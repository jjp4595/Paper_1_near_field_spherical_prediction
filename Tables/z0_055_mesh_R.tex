\begin{table}[htpb]
	\centering
	\caption{Number of cells at ultimate cell length in distance between charge center and perpendicular measuring location at z = 0.055$m.kg^{\frac{1}{3}}$}
\begin{tabularx}{0.4\textwidth}{X X X X}
\toprule
& \multicolumn{3}{l}{\textbf{Zone length (mm)}} \\ 
\multicolumn{1}{r}{\textbf{Resolution Level}}  & \multicolumn{1}{r}{\textbf{50}}   & \multicolumn{1}{r}{\textbf{20}}   & \multicolumn{1}{r}{\textbf{10}}\\ \midrule
\multicolumn{1}{r}{\textbf{0}} & \multicolumn{1}{r}{1}            & \multicolumn{1}{r}{3}            & \multicolumn{1}{r}{5}            \\
\multicolumn{1}{r}{\textbf{1}} & \multicolumn{1}{r}{2}            & \multicolumn{1}{r}{5}            & \multicolumn{1}{r}{10}             \\
\multicolumn{1}{r}{\textbf{2}} & \multicolumn{1}{r}{4}          & \multicolumn{1}{r}{10}             & \multicolumn{1}{r}{20}           \\
\multicolumn{1}{r}{\textbf{3}} & \multicolumn{1}{r}{8}          & \multicolumn{1}{r}{20}           & \multicolumn{1}{r}{40}        \\
\multicolumn{1}{r}{\textbf{4}}& \multicolumn{1}{r}{16}         &               &               \\  
\bottomrule																					  
\end{tabularx}
\end{table}
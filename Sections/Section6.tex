\section{Summary and Conclusions}
This paper presents a methodology for producing a FREM (fast running engineering model) that can predict specific impulse distributions for a given scaled distance. 
The data has been generated by a CFD software, \textit{Apollo blastsimulator} which has been validated by experimental results performed at the University of Sheffield.

Two models have been proposed within this paper, built from a CFD dataset of 100g PE4 charges (with a TNT equivalence factor of 1.2 applied).
The first model can be considered a more accurate FREM for a smaller window of scaled distance, where $i$ represents peak scaled specific impulse:

For $0.10 \leq z \leq 0.20m/kg^{\frac{1}{3}}$ and $0^{\circ} \leq \theta \leq 80^{\circ}$:
\begin{equation}
	i(z, \theta) = 0.322z^{-1.858} \times
	exp \left( \frac{-\left( \frac{\theta}{160} \right) ^2}{2 \times 0.189^2} \right).
\end{equation}

And for $0.10 \leq z \leq 0.52m/kg^{\frac{1}{3}}$ and $0^{\circ} \leq \theta \leq 80^{\circ}$:
\begin{equation}
i(z, \theta) = 0.474z^{-1.663} \times
exp \left( \frac{-\left( \frac{\theta}{160} \right) ^2}{2 \times 0.198^2} \right).
\end{equation}
\section{Introduction}
The blast protection community is equipped with well-established engineering tools, such as the ConWep semi-empirical method \parencite{kinneyExplosiveShocksAir1985} that allows for rapid evaluation of blast wave parameters from a given explosive event. 
These methods are unsuitable, however, when considering explosives located extremely close to a structure, where complex interactions between the explosive and target result in a highly spatially non-uniform loading. 
Experimental work undertaken by \textcite{rigbyExperimentalMeasurementSpecific2019} highlights the complexity of this extreme near-field loading, and outlines the requirement for this complexity to be considered when predicting subsequent structural response \parencite{rigbyPredictingResponsePlates2019}.

There is, therefore, a need for a fast-running model that can provide crucial load characterisation information. This confirmation report presents a machine learning framework that accurately predicts specific impulse distributions for extremely near-field scenarios (within the boundaries of the training database) from a given feature, stand-off.




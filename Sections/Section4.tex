\section{Results}
\subsection{CFD Dataset}
\begin{figure}[htpb]
	\centering
	\includegraphics{Graphs/theta_peak_impulse_0.pdf}
	\caption{Peak specific impulse distribution overview of PE4 dataset.}
	\label{fig:theta_peak_impulse_0}
\end{figure}
The CFD dataset features a cell size of 6.25mm, 100g of PE4 at clear scaled distance of $0.05m/kg^{1/3} \leq z \leq 0.16 m/kg^{1/3}$.


\subsection{Near-field loading model}
\subsubsection{Fitting to peak specific impulse value}
\begin{figure}[htpb]
	\centering
	\includegraphics{Graphs/power_fit.pdf}
	\caption{Fitting a power law to the peak specific impulse data and an assessment of the fit.}
	\label{fig:power_fit}
\end{figure}

\subsubsection{Fitting distribution of theta}
\textbf{Include graph of collapsed CFD values ICr.}

Comparison of l1-norm and l2-norm for Gaussian and effect on total impulse. 

\begin{figure}[htpb]
	\centering
	\includegraphics{Graphs/model_gaussian1.pdf}
	\caption{Gaussian 1 model}
	\label{fig:model_gaussian1}
\end{figure}

\begin{figure}[htpb]
	\centering
	\includegraphics{Graphs/model_henrych.pdf}
	\caption{Henrych/Remennikov model}
	\label{fig:model_henrych}
\end{figure}

\begin{figure}[htpb]
	\centering
	\includegraphics{Graphs/model_RPB_MCEER.pdf}
	\caption{RPB-MCEER model.}
	\label{fig:model_RPB_MCEER}
\end{figure}


Plots in Figures \ref{fig:theta_predictor_residuals_a} and \ref{fig:theta_predictor_residuals_b} show an in depth analyis of residual distributions for Gaussian fits.
If normal probability plot of residuals is approximately linear this supports the condition that error terms are normally distributed. 
%https://online.stat.psu.edu/stat501/lesson/4/4.6

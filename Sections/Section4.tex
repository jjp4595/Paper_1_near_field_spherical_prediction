\section{Results}
\subsection{CFD Dataset}
\begin{figure}[htpb]
	\centering
	\begin{subfigure}[htpb]{0.45\textwidth}
		\centering
		\includegraphics{Graphs/smallz_overview.pdf}
		\caption{$0.05m/kg^{1/3} \leq z \leq 0.16 m/kg^{1/3}$\\$0.05m/kg^{1/3} \leq z\text{(clear standoff)} \leq 0.16 m/kg^{1/3}$}
		\label{fig:smallz_overview}
	\end{subfigure}
	\begin{subfigure}[htpb]{0.45\textwidth}
		\centering
		\includegraphics{Graphs/largez_overview.pdf}
		\caption{$0.05m/kg^{1/3} \leq z \leq 0.5 m/kg^{1/3}$}
		\label{fig:largez_overview}
	\end{subfigure}	
	\begin{subfigure}[htpb]{0.45\textwidth}
	\centering
	\includegraphics{Graphs/data_overview_ex1.pdf}
	\caption{$0.05m/kg^{1/3} \leq z \leq 0.16 m/kg^{1/3}$}
	\label{fig:data_overview_ex1}
\end{subfigure}
\begin{subfigure}[htpb]{0.45\textwidth}
	\centering
	\includegraphics{Graphs/data_overview_ex2.pdf}
	\caption{$0.05m/kg^{1/3} \leq z \leq 0.5 m/kg^{1/3}$}
	\label{fig:data_overview_ex2}
\end{subfigure}	
	\caption{Peak specific impulse distribution overview of PE4 datasets. Scaled distance refers relative to charge centre.}
\end{figure}



\subsection{Near-field loading model}
\subsubsection{Fitting to peak specific impulse value}
\begin{figure}[htpb]
	\centering
	\begin{subfigure}[htpb]{\textwidth}
		\centering
		\includegraphics{Graphs/power_fit_small.pdf}
		\vspace{2mm}
		\caption{$0.05m/kg^{1/3} \leq z \leq 0.16 m/kg^{1/3}$}
		\label{fig:power_fit_small}
	\end{subfigure}
	
	\begin{subfigure}[htpb]{\textwidth}
		\centering
		\includegraphics{Graphs/power_fit_large.pdf}
		\vspace{2mm}
		\caption{$0.05m/kg^{1/3} \leq z \leq 0.5 m/kg^{1/3}$ Tempted to just use this one as it will be more accurate. }
		\label{fig:power_fit_large}
	\end{subfigure}
	\caption{Fitting a power law to the peak specific impulse data  and an assessment of the fit. \textbf{Should this be scaled impulse on the y axis?}}
\end{figure}

\subsubsection{Fitting distribution of theta}


\begin{figure}
	\begin{subfigure}[htpb]{\textwidth}
		\centering
		\includegraphics{Graphs/model_gaussian1.pdf}
		\vspace{3mm}
		\caption{Gaussian 1 model}
		\label{fig:model_gaussian1}
	\end{subfigure}
	
	\begin{subfigure}[htpb]{\textwidth}
		\centering
		\includegraphics{Graphs/model_henrych.pdf}
		\vspace{3mm}
		\caption{Henrych/Remennikov model}
		\label{fig:model_henrych}
	\end{subfigure}
	
	\begin{subfigure}[htpb]{\textwidth}
		\centering
		\includegraphics{Graphs/model_RPB_MCEER.pdf}
		\vspace{3mm}
		\caption{RPB-MCEER model.}
		\label{fig:model_RPB_MCEER}
	\end{subfigure}
	\caption{Normalised specific impulse comparisons for $0.055 < z < 0.16m.kg^{\frac{1}{3}}$.}
\end{figure}

\begin{figure}[htpb]
	\begin{subfigure}{0.45\textwidth}
		\centering
		\includegraphics{Graphs/smallz_gaussian1.pdf}
		\caption{Constructed surface from Gaussian model}
		\label{fig:smallz_gaussian1}
	\end{subfigure}
	\begin{subfigure}{0.45\textwidth}
		\centering
		\includegraphics{Graphs/smallz_gaussian2.pdf}
		\caption{CFD residual}
		\label{fig:smallz_gaussian2}
	\end{subfigure}
\caption{Specific impulse model comparisons for $0.055 < z < 0.16m.kg^{\frac{1}{3}}$.}
\end{figure}


\begin{figure}
	\begin{subfigure}[htpb]{\textwidth}
		\centering
		\includegraphics{Graphs/model_gaussian1_largez.pdf}
		\vspace{3mm}
		\caption{Gaussian 1 model}
		\label{fig:model_gaussian1_largez}
	\end{subfigure}
	\caption{Gaussian model for theta distribution for $0.055 < z < 0.5m.kg^{\frac{1}{3}}$.}
\end{figure}

\begin{figure}[htpb]
	\begin{subfigure}{0.45\textwidth}
		\centering
		\includegraphics{Graphs/largez_gaussian1.pdf}
		\caption{Constructed surface from Gaussian model}
		\label{fig:largez_gaussian1}
	\end{subfigure}
	\begin{subfigure}{0.45\textwidth}
		\centering
		\includegraphics{Graphs/largez_gaussian2.pdf}
		\caption{CFD residual}
		\label{fig:largez_gaussian2}
	\end{subfigure}
	\caption{Specific impulse model comparisons for a large dataset of  $0.055 < z < 0.5m.kg^{\frac{1}{3}}$.}
\end{figure}

If normal probability plot of residuals is approximately linear this supports the condition that error terms are normally distributed. 
%https://online.stat.psu.edu/stat501/lesson/4/4.6

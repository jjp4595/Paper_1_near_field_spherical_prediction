\section{Background}
\subsection{Literature Review}
\textbf{Historical Context}\\
A thorough history of early air-blast experimental work can be found in Esparza \cite{esparzaBlastMeasurementsEquivalency1986}.
This work focuses on developments made from the second world war onwards and the data from these early studies would be compiled into the well established Kingery and Bulmash (KB) semi-empirical predictive methods \cite{kingeryAirblastParametersTNT1984}.
Direct measurements of the blast wave parameters in close proximity to the charge were either \textit{``non-existent or very few''} \cite{esparzaBlastMeasurementsEquivalency1986}, and near-field semi-empirical predictive data were inferred from non direct measurements such as smoke trails \cite{deweyAirVelocityBlast1964} or rudimentary numerical analyses \cite{brodeNumericalSolutionsSpherical1955}.

\textbf{KB Predictive Method}\\
The KB charts are presented in UFC 3-340-02 \cite{dodUnifiedFacilitiesCriteria2008} which are widely used in protective design. 
The utility of the KB charts have been evaluated to provider simpler polynomials \cite{swisdakSimplifiedKingeryAirblast1994}, within 1\% of the original values  but did not include a re-evaluation of the data used to establish the original Polynomials.
Later review sought to compare results of widely used airblast codes, provide experimental comparison and quantify uncertainty in test data \cite{bogosianMeasuringUncertaintyConservatism2002}. 
Results from these experiments suggest that the KB curves ``\textit{substantially underpredicted the recorded data for $Z<2.4m/kg^{1/3}$}''.
Comprehensive reviews of the KB charts and meta-analyses of research reviewing these can be found in \cite{shinAirBlastEffectsCivil2014, karlosAnalysisBlastParameters2016} where updated polynomials to the KB curves are provided.
Reflection coefficients to transform incident to reflected overpressures are reported in \cite{dodUnifiedFacilitiesCriteria2008}, refined further in \cite{shinAirBlastEffectsCivil2014} and validated in \cite{rigbyAngleIncidenceEffects2015}.

Considering medium to large scaled distances specifically, the KB charts have proven to be accurate \cite{rigbyValidationSemiempiricalBlast2014, karlosAnalysisBlastParameters2016}. Equations to calculate the blast wave decay coefficient from the KB data are proposed in \cite{karlosAnalysisBlastWave2016}.

However, uncertainty exists in the exact distributions and magnitudes of extreme near-field loading. 

\textbf{Near-field experimental measurements}\\
One method of near-field experimental measurement involves measuring momentum imparted to small, rigid plugs embedded within a larger target surface \cite{huffingtonReflectedImpulseSpherical1985, nansteelImpulsePlugMeasurements2013}. 
With the limitation here of no gained knowledge of the loading distribution.
An alternative proposed in 1914 by Betram Hopkinson \cite{hopkinsonMethodMeasuringPressure1914} is known as the Hopkinson pressure bar (HPB). 
A length of cylindrical bar that propagates an elastic stress pulse along its axis to be recorded by equipment situated a distance away from the loaded end. 
It is more commonly used in its `split' form \cite{kolskyInvestigationMechanicalProperties1949}, and remain a very useful tool for measuring high-magnitude, short-duration loading \cite{kolskyInvestigationMechanicalProperties1949, edwardsBlastWaveMeasurements1992,piehlerNearFieldImpulseLoading2009,rigbyTestingApparatusSpatial2014,rigbyObservationsPreliminaryExperiments2015,tyasExperimentalStudiesEffect2016, cloeteBlastCharacterizationUsing2016, rigbyNearfieldBlastLoading2019}.
A detailed overview of the experimental set up that generated the data to inform this study is available in \cite{rigbyExperimentalMeasurementSpecific2019}.

\textbf{Near-field CFD}\\
Due to the nature of blast research, datasets can be sparse due to experimental impracticalities (resolution of HPB bar spacing) and costly. 
Computational fluid dynamics (CFD) analyses can be used to complement and add to experimental datasets and are especially useful at providing a higher resolution of data points than HPBs. 
Extensive near-field CFD modelling of TNT spheres has been completed in \cite{shinAirBlastEffectsCivil2014} through the use of a validated 2D CFD code (Autodyn). 
From this modelling work, updated KB polynomials are offered for the range $0.0553<Z<40m/kg^{1/3}$ \cite{shinIncidentNormallyReflected2015}, whilst updated reflection coefficients provided in \cite{shinReflectionCoefficientsReflected2017} for the range $0.16<Z<8m/kg^{1/3}$.
A validation of the CFD code Autodyn in 1D, 2D and 3D for extreme near-field scenarios is summarised in \cite{shinNumericalModelingClosein2014} with comparisons of 1D numerical predictions made against first principles equations offered by Needham \cite{needhamBlastWaves2010}.

\textbf{TNTeq}\\
TNT equivalences are useful factors to enable blast engineers and researchers to compare different explosives in `similar' scenarios. 
$\text{TNT}_{eq}$ factors for common explosives are fairly well established in the far-field, but are not reliable in the near-field \cite{shinTNTEquivalencyOverpressure2015, rigbyInvestigationTNTEquivalence2014, grisaroNumericalInvestigationExplosive2017}.
Hence the model presented here does not allude to any TNT equivalences but instead is built on PE4 data and is able to provide predictions for PE4 only. 

\textbf{NF prediction}\\
CITE MYSELF \cite{pannellPredictingNearFieldSpecific2019}.

\textbf{Final}\\
``\textit{under predict incident and normally reflected peak overpressures and incident impulse near the face of the charge}''
Rigby 2019 need to know the magnitude and distribution to get an idea of the loading. 
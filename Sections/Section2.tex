\section{Background}
\subsection{Literature Review}
\textbf{Historical Context}\\
A thorough history of early air-blast experimental work can be found in Esparza \cite{esparzaBlastMeasurementsEquivalency1986}.
This work focuses on developments made from the second world war onwards and the data from these early studies would be compiled into the well established Kingery and Bulmash (KB) semi-empirical predictive methods \cite{kingeryAirblastParametersTNT1984}.
Direct measurements of the blast wave parameters in close proximity to the charge were either \textit{``non-existent or very few''} \cite{esparzaBlastMeasurementsEquivalency1986}, and near-field semi-empirical predictive data were inferred from non direct measurements such as smoke trails \cite{deweyAirVelocityBlast1964} or rudimentary numerical analyses \cite{brodeNumericalSolutionsSpherical1955}.

\textbf{KB Predictive Method}\\
The KB charts are presented in UFC 3-340-02 \cite{dodUnifiedFacilitiesCriteria2008} which are widely used in protective design. 
The utility of the KB charts have been evaluated to provider simpler polynomials, within 1\% of the original values \cite{swisdakSimplifiedKingeryAirblast1994} but did not include a re-evaluation of the data used to establish the original Polynomials.
Later review sought to compare results of widely used airblast codes, provide experimental comparison and quantify uncertainty in test data \cite{bogosianMeasuringUncertaintyConservatism2002}. 
Results from these experiments suggest that the KB curves ``\textit{substantially underpredicted the recorded data for $Z<2.4m/kg^{1/3}$}''.
Comprehensive reviews of the KB charts and meta-analyses of research reviewing these can be found in \cite{shinAirBlastEffectsCivil2014, karlosAnalysisBlastParameters2016} where updated polynomials to the KB curves are provided.
Reflection coefficients to transform incident to reflected overpressures are reported in \cite{dodUnifiedFacilitiesCriteria2008}, refined further in \cite{shinAirBlastEffectsCivil2014} and validated in \cite{rigbyAngleIncidenceEffects2015}.

Considering medium to large scaled distances specifically, the KB charts have proven to be accurate \cite{rigbyValidationSemiempiricalBlast2014, karlosAnalysisBlastParameters2016}. Equations to calculate the blast wave decay coefficient from the KB data are proposed in \cite{karlosAnalysisBlastWave2016}.

\textbf{Near-field experimental measurements }

\textbf{Final}\\
``\textit{under predict incident and normally reflected peak overpressures and incident impulse near the face of the charge}''
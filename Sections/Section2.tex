\section{Background}
\subsection{Literature Review}
A thorough history of early air-blast experimental work can be found in Esparza \cite{esparzaBlastMeasurementsEquivalency1986}.
This work focuses on developments made from the second world war onwards and the data from these early studies would be compiled into the well established Kingery and Bulmash (KB) semi-empirical predictive methods \cite{kingeryAirblastParametersTNT1984}.
Direct measurements of the blast wave parameters in close proximity to the charge were either \textit{``non-existent or very few''} \cite{esparzaBlastMeasurementsEquivalency1986}, and near-field semi-empirical predictive data were inferred from non direct measurements such as smoke trails \cite{deweyAirVelocityBlast1964} or rudimentary numerical analyses \cite{brodeNumericalSolutionsSpherical1955}.

The KB charts are presented in UFC 3-340-02 \cite{dodUnifiedFacilitiesCriteria2008} which are widely used in protective design. 
In this research the KB charts have been re-evaluated to provider simpler polynomials \cite{swisdakSimplifiedKingeryAirblast1994}, within 1\% of the original values but does not include a re-evaluation of the data used to establish the original polynomials.
Later review sought to compare results of widely used airblast codes, provide experimental comparison and quantify uncertainty in test data \cite{bogosianMeasuringUncertaintyConservatism2002}. 
Results from these experiments suggest that the KB curves ``\textit{substantially underpredicted the recorded data for $z<2.4m/kg^{1/3}$}''.
Comprehensive reviews of the KB charts and meta-analyses of research reviewing these can be found in \cite{shinAirBlastEffectsCivil2014, karlosAnalysisBlastParameters2016} where updated polynomials to the KB curves are provided.
Reflection coefficients that transform incident to reflected overpressures are reported in \cite{dodUnifiedFacilitiesCriteria2008}, refined further in \cite{shinAirBlastEffectsCivil2014} and validated in \cite{rigbyAngleIncidenceEffects2015}.

The KB charts have proven to be accurate for medium to large scaled distances specifically \cite{rigbyValidationSemiempiricalBlast2014, karlosAnalysisBlastParameters2016}, and recent work developing blast wave decay coefficients from the KB data are proposed in \cite{karlosAnalysisBlastWave2016}.
However, uncertainty exists in the exact distributions and magnitudes of near field and extreme near-field loading and is a challenge to the research community.

One method of near-field experimental measurement involves measuring momentum imparted to small, rigid plugs embedded within a larger target surface \cite{huffingtonReflectedImpulseSpherical1985, nansteelImpulsePlugMeasurements2013}. 
With the limitation here of no gained knowledge of the loading distribution.
An alternative proposed in 1914 by Bertram Hopkinson \cite{hopkinsonMethodMeasuringPressure1914} is known as the Hopkinson pressure bar (HPB). 
A length of cylindrical bar that propagates an elastic stress pulse along its axis to be recorded by equipment situated a distance away from the loaded end. 
It is more commonly used in its `split' form, and remain a very useful tool for measuring high-magnitude, short-duration loading \cite{kolskyInvestigationMechanicalProperties1949, edwardsBlastWaveMeasurements1992,piehlerNearFieldImpulseLoading2009,rigbyTestingApparatusSpatial2014,rigbyObservationsPreliminaryExperiments2015,tyasExperimentalStudiesEffect2016, cloeteBlastCharacterizationUsing2016, rigbyNearfieldBlastLoading2019}.
A detailed overview of the experimental set up that generated the data to inform this study is available in \cite{rigbyExperimentalMeasurementSpecific2019}.

Additional experimental set ups include those by \cite{chevalLaboratoryScaleTests2012} with pressure gauges embedded into a modular table to obtain pressure values at various angles of incidence from the charge. 
Experimental set ups such as those in \cite{kleineReflectionBlastWaves2005} present the method of time-resolved omni-directional monochrome schlieren visualisation: a Z-type apparatus with two spherical mirrors and different imaging lenses to visualise the whole flow field.
Newer methods involve the use of high-speed video analysis of explosive blasts, tracking the velocity-radius relationship of the explosive fireball and calculating pressure distributions from this in conjunction with the Rankine-Hugoniot jump conditions \cite{rigbyReflectedNearfieldBlast}.

Due to the nature of blast research, datasets can be sparse due to experimental impracticalities (resolution of HPB bar spacing) and costly. 
Computational fluid dynamics (CFD) analyses can be used to complement and add to experimental datasets and are especially useful at providing a higher resolution of data points than HPBs. 
Near-field CFD modelling in \textit{Apollo blastsimulator} has been undertaken by \cite{whittakerComparisonNumericalAnalysis2019}, with experimental validation from University of Sheffield data. 
Extensive near-field CFD modelling of TNT spheres has been completed in \cite{shinAirBlastEffectsCivil2014} through the use of a validated 2D CFD code (Autodyn). 
From this modelling work, updated KB polynomials are offered for the range $0.0553<Z<40m/kg^{1/3}$ \cite{shinIncidentNormallyReflected2015}, whilst updated reflection coefficients provided in \cite{shinReflectionCoefficientsReflected2017} for the range $0.16<Z<8m/kg^{1/3}$.
A validation of the CFD code Autodyn in 1D, 2D and 3D for extreme near-field scenarios is summarised in \cite{shinNumericalModelingClosein2014} with comparisons of 1D numerical predictions made against first principles equations offered by Needham \cite{needhamBlastWaves2010}.

%As TNT is the standard explosive in the literature, TNT equivalences are useful factors to enable blast engineers and researchers to compare different explosives in `similar' scenarios. 
%These $\text{TNT}_{eq}$ factors for common explosives are fairly well established in the far-field, but are not reliable in the near-field \cite{shinTNTEquivalencyOverpressure2015, rigbyInvestigationTNTEquivalence2014, grisaroNumericalInvestigationExplosive2017}.
%Hence the model presented here does not allude to any TNT equivalences but instead is built on PE4 data and is able to provide predictions for PE4 only. 

Analytical predictions for impulse in near-field explosive events for varying charge shapes is offered by Henrych \cite{henrychDynamicsExplosionIts1979}, and later analytical models appear by Remennikov et al \cite{remennikovExperimentalInvestigationSimplified2017,remennikovExplosiveBreachingWalls2015}. 
A blast load model offered by Randers-Pehrson and Bannister (RPB) \cite{randers-pehrsonAirblastLoadingModel1997} introduced a trigonometric model and applied it to the finite element codes DYNA2d and DYNA3d with a cosine function of the angle of incidence. 
This model also appears in popular commercial FE codes such as LS-DYNA, Abaqus and Autodyn, however the model significantly underestimates the ratios in the vicinity of the transition from regular to mach reflection \cite{schwerAirBlastReflection2017} and can be considered a lower bound representation of reflection ratios when compared to the UFC values \cite{dodUnifiedFacilitiesCriteria2008}.

Further blast load models based on the RPB model are proposed in \cite{jangBlastLoadModel2018,dharmasenaMechanicalResponseMetallic2008}, though the authors do not perform a direct comparison of the loading itself and instead compare them through analysing plate responses to the loading model as a proxy.
The models proposed in \cite{jangBlastLoadModel2018,dharmasenaMechanicalResponseMetallic2008} have not been provided as a comparison in this report.  
Alternative approaches by Pannell et al. \cite{pannellPredictingNearFieldSpecific2019} involve modelling specific impulse distributions using two superposed Gaussian functions and perform non-linear regressions to build a parameter space for a given dataset of CFD results. 
 
Knowledge of air-blast parameters in the extreme near-field are not clear, ``\textit{under predict incident and normally reflected peak overpressures and incident impulse near the face of the charge}'' \cite{shinAirBlastEffectsCivil2014} and current reflection coefficients (for TNT exclusively) have a lower bound of $Z > 0.16m/kg^{1/3}$.
Implementations of the MCEER \cite{shinAirBlastEffectsCivil2014} blast curves have addressed an opportunity to identity precise angle of incidence coefficients for the extreme near-field \cite{wilsonImplementationMCEERTR2018}. 

Rigby et al. \cite{rigbyPredictingResponsePlates2019} propose the concept of ``upper bound kinetic energy uptake'' and demonstrate that in order to obtain any meaningful prediction of plate response, both the magnitude and distribution of the loading must be known.
Therefore there is a pressing need for accurate, fast-running engineering tools that can model specific impulse (and therefore loading) distributions. 
\subsection{Randolph-Pehrson Bannister model with MCEER data}
The specific impulse ($i$) acting on a target is dependent on the angle of incidence of the shock front, theta, with respect to the normal of the target surface. 
From \textcite{randers-pehrsonAirblastLoadingModel1997},
\begin{equation}
i = i_r cos^2\theta + i_{so}(1 + cos^2\theta - 2cos\theta). 
\label{eq:RPB}
\end{equation}
Figure \ref{fig:RPB} shows the angle of incidence and slant distance for a point on a target subjected to a spherical air burst, a distance $R$ away from the target centre.
From trigonometry, we have that $\bar{R}^2 = R^2 + x^2 + y^2$, where $x \text{ and } y$ are horizontal and vertical distance from the centre of the target, and $cos \theta = R / \bar{R}$. 

The RPB model provides the distribution of impulse as a function of incident angle, theta. 
By inputting into the model the peak reflected specific impulse and the peak reflected specific impulse of the slant distance, $\bar{R}$, we can calculate the specific impulse at any given point on a target.
The specific impulse that is fed into the RPB model is obtained from the MCEER-14-0006 curves (Figure \ref{fig:theta_peak_impulse_MCEER_curves}) by \textcite{shinAirBlastEffectsCivil2014}, which are themselves updated versions of the UFC 3-340-02 blast load curves \parencite{dodUnifiedFacilitiesCriteria2008}. 
\begin{figure}[htpb]
	\centering
	\includegraphics{Graphs/theta_peak_impulse_MCEER_curves.pdf}
	\caption{MCEER curves for specific impulse}
	\label{fig:theta_peak_impulse_MCEER_curves}
\end{figure}
\subsection{Henrych / Remennikov analytical model}
From \textcite{henrychDynamicsExplosionIts1979} and Remennikov \cite{remennikovExplosiveBreachingWalls2015}, we have a theoretical basis to calculate specific impulse distributions. 
The hypothesis of a constant velocity for out flowing particles is introduced:
\begin{equation}
u = u_x = \text{ constant}.
\end{equation}
For a spherical charge, it is provided that
\begin{equation}
i = \left(\frac{A_0W}{R^2}\right)cos^2\alpha,
\label{eq:Henrych_sphere}
\end{equation}
where $R$ is the distance from charge centre to centre of target (shown also in Figure \ref{fig:RPB}), $W$ is charge mass. Given
\begin{equation}
A_0 = \frac{N_{xW} + u_x}{4\pi}, 
\end{equation}
and $N_{xW}$ is the displacement velocity of the outburst surface.
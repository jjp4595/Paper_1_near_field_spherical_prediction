\section{Background}
\subsection{Literature Review - NF Blast loading}
Work done recently at the University of Sheffield has demonstrated that to understand structural response, the near field needs to be well defined. This near field loading is not constant in theta due to the non-planar shock front, therefore to accurataly analyse structural response we need to understand the loading distribution.
\parencite{rigbyNearfieldBlastLoading2019, rigbyExperimentalMeasurementSpecific2019, rigbyResponsePlatesNonuniform9, rigbyPredictingResponsePlates2019}


\textcite{shinIncidentNormallyReflected2015} have recommended new polynomials for spherical TNT detonations in free-air for incident and normally reflected peak overpressure and impulse and recommend a minimum cell length of R/500 where R is distance from charge center to measuring location. 

\textcite{shinTNTEquivalencyOverpressure2015} demonstrate that TNT equivalence for differing charges in the near field are not well defined in current methods. Hence this report focuses on PE4 with no TNT equivalence in all scaled calculations (a TNT equivalence of 1).



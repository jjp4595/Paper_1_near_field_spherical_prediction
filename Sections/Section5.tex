\section{Discussion}

\subsection{Comparison of near-field predictive methods for specific impulse}
A comparison of the near-field predictive methods is shown below in Figure \ref{fig:theta_peak_impulse_i_theta_comparisons}. 
The Jang model uses a $\beta$ of 0.8 as proposed by the authors however this coefficient is derived properly by performing finite element analysis. 
The proposed coefficient is used here to demonstrate the overall distribution. 

\begin{figure}[htpb]
	\centering
	\includegraphics{Graphs/theta_peak_impulse_i_theta_comparisons.pdf}
	\caption{Comparison of predictive methods to calculate peak specific impulse.}
	\label{fig:theta_peak_impulse_i_theta_comparisons}
\end{figure}

Due to TNT equivalence issues, the left plot displaying absolute values should be considered cautiously. 
A more useful interpretation is given in the plot on the right hand side, that shows the impulse as a ratio of the maximum (at $\theta=0$). 
From this plot, it is shown that the Henrych model and the RPB models mimic the shape of the CFD output and experimental results in a more accurate way than the Jang and Dharmasena model. 
However, the RPB and Henrych models do display a faster rate of decrease between 20 and 60 degrees, when integrated around 360 degrees this will have a profound effect on the overall impulse and hence are not extremely accurate representations of the impulse distributions.   
The Henrych and RPB models are taken forward for further analysis. 


\subsection{Area integrated impulse comparison}
\begin{figure}[htpb]
	\centering
	\begin{subfigure}{\linewidth}
		\centering
		\includegraphics{Graphs/theta_peak_impulse_i_percent_a.pdf}
		\caption{RPB-MCEER}
		\label{fig:theta_peak_impulse_i_percent_a}
	\end{subfigure}
	\begin{subfigure}{\linewidth}
		\centering
		\includegraphics{Graphs/theta_peak_impulse_i_percent_b.pdf}
		\caption{Henrych}
		\label{fig:theta_peak_impulse_i_percent_b}
	\end{subfigure}
	\begin{subfigure}{\linewidth}
		\centering
		\includegraphics{Graphs/theta_peak_impulse_i_percent_c.pdf}
		\caption{CFD}
		\label{fig:theta_peak_impulse_i_percent_c}
	\end{subfigure}
	\caption{}
	\label{fig:theta_peak_impulse_i_percent_}
\end{figure}
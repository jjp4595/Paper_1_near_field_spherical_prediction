\section{Evaluation}
\subsection{Specific impulse evaluation}
From the presented model, the user now has access to the peak scaled specific impulse and the distribution of the peak scaled specific impulse for a given scaled distance.
These two equations are all that is required to reconstruct the data shown in Figures \ref{fig:data_overview_small_a} and \ref{fig:data_overview_large_a}.
For the small dataset, Figure \ref{fig:smallz_gaussian1} demonstrates this reconstructed surface whilst Figure \ref{fig:smallz_gaussian2} presents the difference between the reconstructed and original surface. 
It is shown to reproduce the surface to a high level of accuracy.
Similarly, Figures \ref{fig:largez_gaussian1} and \ref{fig:largez_gaussian2} present the model evaluation for the large dataset.
Considering this has made a minor sacrifice for accuracy to enable a larger range of predictions, the accuracy remains high.
\begin{figure}[htpb]
	\begin{subfigure}{0.45\textwidth}
		\centering
		\includegraphics{Graphs/smallz_gaussian1.pdf}
		\caption{Constructed surface from Gaussian model}
		\label{fig:smallz_gaussian1}
	\end{subfigure}
	\begin{subfigure}{0.45\textwidth}
		\centering
		\includegraphics{Graphs/smallz_gaussian2.pdf}
		\caption{CFD residual}
		\label{fig:smallz_gaussian2}
	\end{subfigure}
	\caption{Specific impulse model comparisons for $0.1 \leq z \leq 0.2m/kg^{\frac{1}{3}}$.}
	\label{fig:smallz_gaussians}
\end{figure}

\begin{figure}[htpb]
	\begin{subfigure}{0.45\textwidth}
		\centering
		\includegraphics{Graphs/largez_gaussian1.pdf}
		\caption{Constructed surface from Gaussian model}
		\label{fig:largez_gaussian1}
	\end{subfigure}
	\begin{subfigure}{0.45\textwidth}
		\centering
		\includegraphics{Graphs/largez_gaussian2.pdf}
		\caption{CFD residual}
		\label{fig:largez_gaussian2}
	\end{subfigure}
	\caption{Specific impulse model comparisons for a large dataset of  $0.1 \leq z \leq 0.52m/kg^{\frac{1}{3}}$.}
\end{figure}

\subsection{Area integrated total impulse comparison}
A final evaluation of the two models was performed via a total impulse comparison on a $5m$ diameter circular target plate.
Figure \ref{fig:smallz_totalimpulse_gauss} presents a constructed total scaled impulse surface, by varying the scaled radius (therefore changing the size of plate) and scaled distance.
The same total scaled impulse calculation was performed on the CFD data itself, and subtracted from the predicted model to provide an indication of accuracy. 
Up to a scaled radius of $3m/kg^{\frac{1}{3}}$ predictions within $10\%$ are shown, with the model performing comparatively weaker at larger values of scaled radius.
This has been illustrated previously where the model is slightly weaker predicting results from approximately 50 degrees onwards.

For the larger dataset we see a similar pattern, in that smaller scaled distances and a scaled radius of $3m/kg^{\frac{1}{3}}$ onwards provide a weaker prediction.
The majority of both surfaces however provide good approximations for total scaled impulse, with only the larger angle of incidence values (ie larger scaled radius) and smaller scaled distances providing a weaker prediction.
\begin{figure}[htpb]
	\begin{subfigure}{0.45\textwidth}
		\centering
		\includegraphics{Graphs/smallz_totalimpulse_gauss.pdf}
		\caption{Proposed model surface}
		\label{fig:smallz_totalimpulse_gauss}
	\end{subfigure}
	\begin{subfigure}{0.45\textwidth}
		\centering
		\includegraphics{Graphs/smallz_totalimpulse_gauss2.pdf}
		\caption{Model residual}
		\label{fig:smallz_totalimpulse_gauss2}
	\end{subfigure}
	\caption{Total impulse on a circular target of various radii for scaled distance of $0.1 \leq z \leq 0.2m/kg^{\frac{1}{3}}$}
\end{figure}

\begin{figure}[htpb]
	\begin{subfigure}{0.45\textwidth}
		\centering
		\includegraphics{Graphs/largez_totalimpulse_gauss.pdf}
		\caption{Constructed surface from Gaussian model}
		\label{fig:largez_totalimpulse_gauss}
	\end{subfigure}
	\begin{subfigure}{0.45\textwidth}
		\centering
		\includegraphics{Graphs/largez_totalimpulse_gauss2.pdf}
		\caption{Model residual}
		\label{fig:largez_totalimpulse_gauss2}
	\end{subfigure}
	\caption{Total impulse on a circular target of various radii for scaled distance of $0.1 \leq z \leq 0.52m/kg^{\frac{1}{3}}$}
\end{figure}
\section{Air blast loading using \textit{Apollo Blastsimulator}}
\subsection{Mesh Sensitivity}

\import{Tables/}{z0_055_mesh_mm.tex}
\import{Tables/}{z0_055_mesh_R.tex}
\import{Tables/}{z0_5_mesh_mm.tex}
\import{Tables/}{z0_5_mesh_R.tex}
Include theta vs I graph with multiple cell resolutions and experimental values. 
Can discuss that convergence takes longer around areas of mach stem formation etc. 
Apollo may appear to converge sooner due to the staging algorithm it uses. 
The R/cell length is actually a lot smaller because of the smaller stage length but consistent resolution.

\begin{figure}[htpb]
	 \begin{subfigure}{\linewidth}
	 	\centering
	 	\includegraphics{Graphs/mesh_convergence_z0_055_1.pdf}
	 	\caption{z = 0.055$m.kg^{\frac{1}{3}}$}
	 	\label{fig:mesh_convergence_z0_055_1}
	\end{subfigure}
	\begin{subfigure}{\linewidth}
 		\centering
 		\includegraphics{Graphs/mesh_convergence_z0_5_1.pdf}
 		\caption{z = 0.5$m.kg^{\frac{1}{3}}$}
 		\label{fig:mesh_convergence_z0_5_1}
 	\end{subfigure}
	\caption{Mesh convergence study for peak impulse at various ultimate cell lengths.}
	\label{fig:mesh_convergence}
\end{figure}


\begin{figure}[htpb]
	\begin{subfigure}{\linewidth}
		\centering
		\includegraphics{Graphs/mesh_convergence_z0_055_2.pdf}
		\caption{0 degrees}
		\label{fig:mesh_convergence_z0_055_2}
	\end{subfigure}
	\begin{subfigure}{\linewidth}
		\centering
		\includegraphics{Graphs/mesh_convergence_z0_055_2a.pdf}
		\caption{20 degrees}
		\label{fig:mesh_convergence_z0_055_2a}
	\end{subfigure}
	\begin{subfigure}{\linewidth}
		\centering
		\includegraphics{Graphs/mesh_convergence_z0_055_2b.pdf}
		\caption{40 degrees}
		\label{fig:mesh_convergence_z0_055_2b}
	\end{subfigure}
	\begin{subfigure}{\linewidth}
		\centering
		\includegraphics{Graphs/mesh_convergence_z0_055_2c.pdf}
		\caption{60 degrees}
		\label{fig:mesh_convergence_z0_055_2c}
	\end{subfigure}
	\begin{subfigure}{\linewidth}
		\centering
		\includegraphics{Graphs/mesh_convergence_z0_055_2d.pdf}
		\caption{80 degrees}
		\label{fig:mesh_convergence_z0_055_2d}
	\end{subfigure}
	\caption{Overpressure and impulse time histories at z=0.055$m.kg^{\frac{1}{3}}$ for various angles of incidence.}
	\label{fig:mesh_convergence_angle_zlow}
\end{figure}

\begin{figure}[htpb]
	\begin{subfigure}{\linewidth}
		\centering
		\includegraphics{Graphs/mesh_convergence_z0_5_2.pdf}
		\caption{0 degrees}
		\label{fig:mesh_convergence_z0_5_2}
	\end{subfigure}
	\begin{subfigure}{\linewidth}
		\centering
		\includegraphics{Graphs/mesh_convergence_z0_5_2a.pdf}
		\caption{20 degrees}
		\label{fig:mesh_convergence_z0_5_2a}
	\end{subfigure}
	\begin{subfigure}{\linewidth}
		\centering
		\includegraphics{Graphs/mesh_convergence_z0_5_2b.pdf}
		\caption{40 degrees}
		\label{fig:mesh_convergence_z0_5_2b}
	\end{subfigure}
	\begin{subfigure}{\linewidth}
		\centering
		\includegraphics{Graphs/mesh_convergence_z0_5_2c.pdf}
		\caption{60 degrees}
		\label{fig:mesh_convergence_z0_5_2c}
	\end{subfigure}
	\begin{subfigure}{\linewidth}
		\centering
		\includegraphics{Graphs/mesh_convergence_z0_5_2d.pdf}
		\caption{80 degrees}
		\label{fig:mesh_convergence_z0_5_2d}
	\end{subfigure}
	\caption{Overpressure and impulse time histories at z=0.5$m.kg^{\frac{1}{3}}$ for various angles of incidence.}
	\label{fig:mesh_convergence_angle_zhigh}
\end{figure}

\FloatBarrier
\subsection{Validation against experimental data}
80mm NF data
380mm NF data


\section{Air blast loading using \textit{Apollo Blastsimulator}}
\subsection{Mesh Sensitivity}

\begin{table}[htpb]
	\centering
	\caption{Ultimate cell length for mesh convergence study at z = 0.055$m.kg^{\frac{1}{3}}$}
	\import{Tables/}{z0_055_mesh_mm.tex}
\end{table}
\begin{table}[htpb]
	\centering
	\caption{Number of cells at ultimate cell length in distance between charge center and perpendicular measuring location at z = 0.055$m.kg^{\frac{1}{3}}$}	
	\import{Tables/}{z0_055_mesh_R.tex}
\end{table}

Include theta vs I graph with multiple cell resolutions and experimental values. 
Can discuss that convergence takes longer around areas of mach stem formation etc. 
Apollo may appear to converge sooner due to the staging algorithm it uses. 
The R/cell length is actually a lot smaller because of the smaller stage length but consistent resolution.
Quicker convergence for total impulse. 

\begin{figure}[htpb]
	 \begin{subfigure}{\linewidth}
	 	\centering
	 	\includegraphics{Graphs/mesh_convergence_z0_055_1.pdf}
	 	\caption{z = 0.055$m.kg^{\frac{1}{3}}$}
	 	\label{fig:mesh_convergence_z0_055_1}
	\end{subfigure}
\end{figure}


\begin{figure}[htpb]
	\begin{subfigure}{\linewidth}
		\centering
		\includegraphics{Graphs/mesh_convergence_z0_055_2.pdf}
		\caption{0 degrees}
		\label{fig:mesh_convergence_z0_055_2}
	\end{subfigure}
	\begin{subfigure}{\linewidth}
		\centering
		\includegraphics{Graphs/mesh_convergence_z0_055_2a.pdf}
		\caption{20 degrees}
		\label{fig:mesh_convergence_z0_055_2a}
	\end{subfigure}
	\begin{subfigure}{\linewidth}
		\centering
		\includegraphics{Graphs/mesh_convergence_z0_055_2b.pdf}
		\caption{40 degrees}
		\label{fig:mesh_convergence_z0_055_2b}
	\end{subfigure}
	\begin{subfigure}{\linewidth}
		\centering
		\includegraphics{Graphs/mesh_convergence_z0_055_2c.pdf}
		\caption{60 degrees}
		\label{fig:mesh_convergence_z0_055_2c}
	\end{subfigure}
	\begin{subfigure}{\linewidth}
		\centering
		\includegraphics{Graphs/mesh_convergence_z0_055_2d.pdf}
		\caption{80 degrees}
		\label{fig:mesh_convergence_z0_055_2d}
	\end{subfigure}
	\caption{Overpressure and impulse time histories at z=0.055$m.kg^{\frac{1}{3}}$ for various angles of incidence.}
	\label{fig:mesh_convergence_angle_zlow}
\end{figure}

\begin{table}[htpb]
	\centering
	\caption{Information for mesh strategy in context of entire range of clear scaled distance ($0.055 < z < 0.16m.kg^{\frac{1}{3}}$).}	
	\import{Tables/}{dataset_meshes.tex}
\end{table}

\begin{figure}[htpb]
	\begin{subfigure}{0.45\textwidth}
		\centering
		\includegraphics{Graphs/mesh_convergence_dataset1.pdf}
		\caption{Specific impulse distribution overview for $0.055 < z < 0.16m.kg^{\frac{1}{3}}$.}
		\label{fig:mesh_convergence_dataset1}
	\end{subfigure}
	\begin{subfigure}{0.45\textwidth}
	\centering
	\includegraphics{Graphs/mesh_convergence_dataset2.pdf}
	\caption{Specific impulse distribution for z $= 0.133m.kg^{\frac{1}{3}}$ for DMA analysis.}
	\label{fig:mesh_convergence_dataset2}
	\end{subfigure}
	\begin{subfigure}{0.45\textwidth}
	\centering
	\includegraphics{Graphs/mesh_convergence_dataset3.pdf}
	\caption{Total area integrated impulse impinging on a $1m^2$ target plate for each mesh.}
	\label{fig:mesh_convergence_dataset3}
\end{subfigure}	
\end{figure}



\FloatBarrier
\subsection{Validation against experimental data}
This section takes forward a selection of meshes to validate against experimental results.

\begin{figure}[htpb]
	\centering
	\includegraphics{Graphs/mesh_convergence_z0_055_3.pdf}
	\caption{Validation of mesh for investigating specific impulse distributions at z = 0.055$m.kg^{\frac{1}{3}}$ and z = 0.12$m.kg^{\frac{1}{3}}$.}
	\label{fig:mesh_convergence_z0_055_3}
\end{figure}
Figure \ref{fig:mesh_convergence_z0_055_3} shows the peak specific impulse distribution for the lower bound of the dataset, whilst also showing the results from experimental work with a larger scaled distance.
Though the peak impulse (left-side plot) is considerably lower as the scaled distance increases, the ratio of peak specific impulse compared to the maximum (at theta $= 0$) remains constant in theta. 
This result is interesting as it suggests the peak specific impulse distribution (as a proportion of its maximum at theta $=0$) is invariant to scaled distance between these bounds. 
The point in which this no longer holds true, however, will require further research. 

\begin{figure}[htpb]
	\begin{subfigure}{\linewidth}
		\centering
		\includegraphics{Graphs/80mm_validation.pdf}
		\caption{0 degrees}
		\label{fig:80mm_validation}
	\end{subfigure}
	\begin{subfigure}{\linewidth}
		\centering
		\includegraphics{Graphs/80mm_validation_a.pdf}
		\caption{17 degrees}
		\label{fig:80mm_validation_a}
	\end{subfigure}
	\begin{subfigure}{\linewidth}
		\centering
		\includegraphics{Graphs/80mm_validation_b.pdf}
		\caption{32 degrees}
		\label{fig:80mm_validation_b}
	\end{subfigure}
	\begin{subfigure}{\linewidth}
		\centering
		\includegraphics{Graphs/80mm_validation_c.pdf}
		\caption{43 degrees}
		\label{fig:80mm_validation_c}
	\end{subfigure}
	\begin{subfigure}{\linewidth}
		\centering
		\includegraphics{Graphs/80mm_validation_d.pdf}
		\caption{51 degrees}
		\label{fig:80mm_validation_d}
	\end{subfigure}
	\caption{Experimental validation for overpressure and impulse time histories at z=0.12 $m.kg^{\frac{1}{3}}$ for various angles of incidence.}
	\label{fig:80mm_validation_OP_imp}
\end{figure}
\FloatBarrier

